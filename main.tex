\documentclass[12pt, answers]{exam}
%\documentclass[12pt]{exam}
\usepackage[utf8]{inputenc}
\usepackage{soul}
\usepackage{bm}
\usepackage{listings}
\usepackage{pgfplots} 
%\usepackage[marginparsep=16pt, margin=1.75in, showframe]{geometry}
\usepackage[marginparsep=16pt, margin=1.0in]{geometry}

\usepackage{amsmath,amssymb}
\usepackage{multicol}
\usepackage{marginnote}

\newcommand{\class}{PHDCW-101}
\newcommand{\term}{2019 Semester}
\newcommand{\examnum}{End-Semester Examination}
\newcommand{\examdate}{\today}
\newcommand{\timelimit}{60 Minutes}

\pagestyle{head}
\firstpageheader{}{}{}
\runningheader{\class}{\examnum\ - Page \thepage\ of \numpages}{\examdate}
\runningheadrule
\DeclareMathOperator{\sech}{sech}
\begin{document}

\noindent
\begin{tabular*}{\textwidth}{l @{\extracolsep{\fill}} r @{\extracolsep{6pt}} l}
\textbf{\class} & \\
\textbf{\term} &&\\
\textbf{\examnum} &&\\
\textbf{\examdate} &&\\
\textbf{Time Limit: \timelimit} & Instructor: Dr. Analabha Roy
\end{tabular*}\\
\rule[2ex]{\textwidth}{2pt}

\begin{questions}

\question\marginnote{$10$} Answer any $5$ questions.
\begin{parts}
\part\marginnote{$2$}  What do you mean by precise and accurate measurements? 
\part\marginnote{$2$}Power in a circuit is measured by measuring a current through a resistor. The current is measured with an accuracy of $\pm 1.5\% $ and the tolerance band of the resistor is $\pm 0.5\%$. The errors are limiting or guarantee errors. Find the accuracy with which the power is measured.
\part\marginnote{$2$}  What will be the output of the following python program?
\begin{lstlisting}[language=Python]
import numpy as np
e  = np.array([(1,2,3), (4,5,6)])
print(e)
e.reshape(3,2)
print(e)
\end{lstlisting}
\part\marginnote{$2$}  Write LaTeX expressions for rendering the following equations.
\begin{subparts}
\subpart
\begin{equation}
\int^\infty_0\mathrm{d}x\;\frac{\sin{x}}{x} = \frac{\pi}{2}.\nonumber
\end{equation}
\subpart
\begin{equation}
\sum^N_{n=0} r^n = \frac{1-r^N}{1-r}.\nonumber
\end{equation}
\end{subparts}
\part\marginnote{$2$} 
\part\marginnote{$2$} 
\part\marginnote{$2$} 
\end{parts}

\question\marginnote{$40$}  Answer any $4$ questions.

\begin{parts}
\part\marginnote{$10$}  Answer the following questions.
\addpoints
\begin{subparts}
\subpart\marginnote{$2$} 
\subpart\marginnote{$4$}  
\subpart\marginnote{$4$} 
\end{subparts}
\part\marginnote{$10$}  Answer the following questions.
\begin{subparts}
\subpart\marginnote{$2$} 
\subpart\marginnote{$2$} 
\subpart\marginnote{$3$} 
\subpart\marginnote{$3$} 
\end{subparts}
\part\marginnote{$10$} 
Write down the output of the following \LaTeX \ snippet.
\begin{lstlisting}[language=TeX]
\documentclass [12pt,a4] {article}

\usepackage{amsmath}

\title{{\Large	\bf Chaos in the Quantum Double Well Oscillator:\\
		The Ehrenfest View Revisited}}

\begin{document}
\maketitle

It is generally agreed that the full quantum dynamics does not exhibit 
chaos. For systems which exhibit chaotic dynamics in the classical
limit, it was clearly established by Fishman, Grempel and Prange
that there exists a critical time $t \sim 0$ ($1/h^{1/r}$) beyond
which the dynamics crosses over to the quantal behaviour.  It was 
conjectured in the late eighties that there could be systems where 
the classical dynamics is obviously regular but the semiquantal 
dynamics can be chaotic. In support of this conjecture,
Pattanayak and Schieve explored the semiquantal dynamics of the
double well oscillator governed by the Hamiltonian
\begin{equation}
H = \frac{P^2} {2} - \frac{1} {2} \ x^2 + \frac{\lambda} {4} \ x^4
\end{equation}
The classical dynamics of this oscillator is obviously regular, as
explained in Landau-Lifshitz (Vol. 1), being a periodic trajectory
centered about $x  
= \pm {1\over \sqrt{\lambda}}$ for total energy $E$ in the range 
$0 > E > -1/4$ and about $x=0$ for $E>0$. It was shown, based on an
Ehrenfest equation approach that in the semiquantal limit the quantum
fluctuations cause the dynamics of this oscillator to become chaotic
with four repelling zones in the phase space.  

The semiquantal 
Ehrenfest dynamics of the centroid $\langle x\rangle$ of a wave packet is given by
\begin{equation}
\langle \ddot x\rangle = \langle x\rangle - \lambda \langle x^3\rangle ,
\end{equation}
\textit{i.e.}
\begin{equation}
\langle\ddot x\rangle - \langle x\rangle + \lambda \langle x\rangle^3 
= \lambda \left[\langle x\rangle^3 - \langle x^3\rangle\right] = Q
\end{equation}
With this in mind, the class of quantum wave packets that we
will work with is
\begin{equation}
\psi (x,t) = N_1 (t) e^{- \left(x-a_0 - \epsilon (t)\right)^2/2 b^2_0} + 
N_2 (t) e^{- \left(x+a_0 + \epsilon (t)\right)^2/2 b^2_0} .
\end{equation}

Turning to the dynamics, we now have the result
\begin{subequations}
\begin{equation}
\langle x\rangle = (a_0 + \epsilon) (N_1^2 - N_2^2) , 
\end{equation}
\begin{equation}
\langle x^3\rangle = (a_0 + \epsilon) \left[(a_0 + \epsilon)^2 
+ \frac{3} {2} b^2_0\right] (N_1^2 - N^2_2) .
\end{equation}
\end{subequations}

\end{document}
\end{lstlisting}
\pagebreak
\part\marginnote{$10$}  Answer the following questions.
\begin{subparts}
\subpart\marginnote{$4$}  Name and describe four interfaces to the Unix/Linux kernel.
\subpart\marginnote{$4$}  Briefly describe the four stages of the building processes of a C program.
\subpart\marginnote{$2$}  What are the differences between a high-level and mid-level programming language?
\end{subparts}
\part\marginnote{$10$}  Answer the following questions
\begin{subparts}
\subpart\marginnote{$4$} Write down the statement of theory of least squares. An observer measures a physical quantity as $1.70, 1.78, 1.74, 1.79$, and $1.74$. Apply the theory of least squares to find the most probable
value.
\subpart\marginnote{$3$} Applying theory of least squares, set up the normal equations for a straight line $y=a+bx$.
\subpart\marginnote{$3$}Two resistances $100\Omega \pm 5\Omega $ and $150\Omega \pm 15\Omega $ are connected in series. If the deviations are standard deviations, how the resultant resistance can be expressed? 
\end{subparts}
\part\marginnote{$10$}  Answer the following questions.
\begin{subparts}
\subpart\marginnote{$5$}Write down the normal distribution function explaining the symbols therein. Show that $\sigma^2 = \overline{x^2}-\bar{x}^2 $, where the symbols have their usual meaning. 
\subpart\marginnote{$3$}A box contains $100 \Omega $ resistors which are known to have a standard deviation of $2 \Omega$. What is the probability of finding a resistor in the range $99-101 \;\Omega$ ? 
\subpart\marginnote{$2$} Consider error propagation through a single-variable function $Z = f(A)$.
Here, $\alpha_A$ represents the error on the mean $\bar{A}$ . Show that the uncertainty in $Z$ can also be represented as $\alpha_Z = \left|\displaystyle\frac{\mathrm{d}Z}{\mathrm{d}A}\right|\alpha_A$.
\end{subparts}
\part\marginnote{$10$}  Answer the following questions.
\begin{subparts}
\subpart\marginnote{$3$} 
\subpart\marginnote{$3$} 
\subpart\marginnote{$4$} 
\end{subparts}
 
\end{parts}


\end{questions}

\end{document}